\documentclass{article}
\usepackage{amsmath, amsthm, amsfonts, amssymb}
\usepackage{bbm}
\usepackage{hyperref}
\usepackage{cleveref}
\usepackage{mathrsfs}
\usepackage{listings}
\usepackage{color}

\newcommand\numberthis{\addtocounter{equation}{1}\tag{\theequation}}

\theoremstyle{definition}
\newtheorem{proposition}{Proposition}
\DeclareMathOperator\supp{supp}


\title{Credit \& Credibility }
\author{Arnav Sood (\texttt{arnav.sood@ubc.ca}) \\ \normalsize Supervised by: Prof. Vitor Farinha Luz (\texttt{vitor.farinhaluz@ubc.ca})}
\date{\today}

\begin{document}
\maketitle

\paragraph*{Overivew} We study a model of strategic behavior by a profit-maximizing credit rating agency (CRA). The rater jointly chooses the ratings and the (flat) price of a rating, so as to influence the size and composition of the set of ratings-seeking firms. We derive equilibrium conditions for the static case, and present some numerical results. 

Of course, this is not a \emph{sui generis} area of research. Bolton, Freixas, and Shapiro (2010) model CRAs as players in a competitive market. Bar-Isaac and Shapiro (2010) analyze the time-varying nature of their incentives and behavior over the business cycle, and Kamenica and Gentzow (2011) establish the seminal ``Bayesian Persuasion" setting for strategic information transmission. 

On the empirical side, Gillette et al. (2018) find that a 2010 upward shock to ratings (without a corresponding improvement in fundamentals) ``led to larger fees and to increases in [CRA] market share.'' Xia (2011) concludes something similar from a dataset of over 70,000 U.S. credit ratings.

We are indebted to all of this work (and more). The chief contribution of this note is to jointly solve the pricing problem and ``decision problem'' for ratings in one shot.

\paragraph*{JEL Codes} G24, D82, D83

\newpage

\tableofcontents

\newpage

\section{Setting}

\subsection{Agents}

The model has three classes of agent: 

\begin{itemize}
    \item A mass-one continuum of firms, who have types $\theta_i$ distributed i.i.d. in $\{H, L\}$ (``High,'' ``Low.'') Each firm also has an associated \emph{outside option} $\omega_i$, which is drawn i.i.d. from a (differentiable, etc.) type-contingent distribution $F_\theta$ with compact support $[\underline{\omega_\theta}, \overline{\omega_\theta}]$
    
    Firms solve a static decision problem about market entry: 

    \begin{equation}
        V_i \equiv \max \{\omega_i, \mathbb{E} U(\theta) - p \}
        \label{firm-objective}
    \end{equation}

    where $U(\theta)$ is the payoff from obtaining a rating as a firm of type $\theta$, and $p$ is the price of a rating.
    
    \item A single \emph{credit rater} (CRA). The CRA's goal is to choose a ratings scheme $m$ (in practice, summarized by a dishonesty parameter $\phi$ and a price $p$), to solve 
    
    \begin{equation}
        \max\limits_{p \geq 0, \phi \in [0, 1]} p \{F_L(\phi R - p)(1 - \lambda) + F_H(R - p)\lambda\}
        \label{CRA-objective}
    \end{equation}

    In \eqref{CRA-objective}, $\lambda$ is the common prior about the share of $H$-type firms.

    \item A \emph{public debt market}, which maps signals to investment outcomes according to a simple cutoff rule: 
    
    \begin{equation}
        \label{market-objective}
        U: \theta \mapsto R \mathbbm{1}_{\lambda^o(\theta) \geq \underline{\lambda}, \theta = h}
    \end{equation}

    That is: if the market's belief that a firm in question is of type $H$ (written $\lambda^o(h)$) exceeds some exogenous cutoff ($\underline{\lambda}$), then the firm is assigned a flat investment $R \in \text{supp}F_H$.  

    Note that we've implicitly assumed the set of firm signals is $\{h, l\}$ such that $m: H \mapsto h$ always. We will defend this later. 
\end{itemize}

\subsection{Timing}

\begin{enumerate}
    \item The CRA chooses a scheme $m \equiv (p, \phi)$ according to the common prior $\lambda$ about the share of high-type firms, in order to solve \eqref{CRA-objective}.
    
    \item Firms observe $(p, \phi)$, and make entry decisions according to \eqref{firm-objective}. Denote the proportion of firms of type $\theta$ which decide to buy ratings as $e_\theta$ (i.e., the mass of high entrants is $e_H \lambda$.)
    
    \item The market observes $\{(p, \phi), e_H, e_L\}$, and mechanistically executes the investment rule in \eqref{market-objective}.
\end{enumerate}

\subsection{Caveats}

Before proceeding with an analysis of this model, it's worth examining its implicit assumptions. 

\begin{enumerate}
    \item \textbf{No Transfer Payments:} It's conceivable that firms of the low type (who may have more to gain from a high credit rating, say if $\mathbb{E} F_L < \mathbb{E} F_H$) might pay more to the CRA for a good rating. This is ruled out by our flat price $p$.
    
    \item \textbf{No Outside Information:} In this simple model, the only source of information available to the market is the single CRA. In practice, there is a competitive market for information\footnote{See, e.g., Veldkamp (2005)}, which may discipline the information put out by a CRA.
    
    \item \textbf{No Inside Information:} Likewise, we do not allow firms to generate their own information (e.g., in reality a high-type firm may decide to seek debt financing instead of equity financing, so as to signal its type.)
    
    \item \textbf{No CRA Competition:} In practice, the credit-ratings industry is an oligopoly and not a monopoly (i.e., it is dominated by Fitch, Moody's, and Standard \& Poor's.) So the monopoly pricing section of this note is likely not generalizable.
\end{enumerate}

Of course, there are many other unrealistic aspects of this simple static model, but these are perhaps the most significant.

\newpage 

\section{General Case}

In this section, we derive objects such as equilibrium conditions which are relevant to most cases of the model (i.e., we impose relatively loose restrictions on parameters.)

\subsection{Equilibrium Conditions}

\subsubsection{Market's Problem}

Begin with the market's problem; implementing the investment rule \eqref{market-objective}. The key is to construct an expression for $\lambda^o(h)$, or market confidence in the validity of an $h$ signal.

First, define the \emph{inflow prior}, or the confidence that $\theta = H$ conditional merely on seeking a rating. 

\begin{equation}
    \label{inflow-prior}
    \lambda^I \equiv \frac{e_H \lambda}{e_H \lambda + e_L (1 - \lambda)}
\end{equation}

This is the raw proportion of high types in the ratings pool, ignoring the actual ratings.

Using this object, and we can give an expression for:

\begin{equation}
    \label{lambda-o}
    \lambda^o(h) \equiv \Pr[\theta = H | h]
\end{equation}

Which is: 

\begin{equation}
    \lambda^o(h) = \frac{\lambda^I}{\lambda^I + (1 - \lambda^I)\phi}
\end{equation}

Note that (as mentioned earlier, and to be justified later) $m : H \mapsto h$ always, so dishonesty is strictly about ratings $L \mapsto h$. These happen at rate $\phi$.

After substitution: 

\begin{equation}
    \label{lambda-o-final}
    \lambda^o(h) = \frac{e_H \lambda}{e_H \lambda + \phi e_L (1 - \lambda)}
\end{equation}

\subsubsection{Firms' Problem}

As above, firms are mechanically making entry decisions based on \eqref{firm-objective}. 

Assume that $\lambda^o(h) \geq \underline{\lambda}$ (otherwise, all firms would face market return 0.) Assume also that there is an interior solution (so $F_\theta \in (0, 1)$.) Then, we have: 

\begin{align}
    e_H &= F_H(R - p) \label{high-entry} \\ 
    e_L &= F_L(\phi R - p) \label{low-entry}
\end{align}

In other words, given continuua of firms, the share of firms with outside options leading to entry is exactly given by the CDF of the relevant distribution.

\subsubsection{Rater's Problem: Credibility Constraint}

The problem here is to solve \eqref{CRA-objective} in a forward-looking way (i.e., anticipating \eqref{high-entry}, \eqref{low-entry}, \eqref{lambda-o-final}.)

However, there is a constraint. To motivate entry, we assumed that $\lambda^o(h) \geq \underline{\lambda}$ (that is, the market is confident enough in the $h$ signal to invest.) That is: 

\begin{align*}
    \lambda^o(h) &\geq \underline{\lambda} \\ 
    \frac{e_H \lambda}{e_H \lambda + \phi e_L (1 - \lambda)} &\geq \underline{\lambda} \hspace{5 em} \text{by \eqref{lambda-o-final}} \\ 
    e_H \lambda (1 - \underline{\lambda}) &\geq \phi e_L \underline{\lambda} (1 - \lambda) \\ 
    \phi &\leq \frac{e_H \lambda (1 - \underline{\lambda})}{e_L \underline{\lambda} (1 - \lambda)} \label{MC-cons}\numberthis
\end{align*}

This last constraint (which we call the \textbf{Market Credibility} constraint, or (MC)) can be interpreted as a sort of discipline on the CRA's dishonesty $\phi$. It says that the CRA cannot lie so much that the market ceases to invest in any firms. 

\subsubsection{Raters' Problem: Equilibrium Conditions}

We wish to solve \eqref{CRA-objective} subject to \eqref{MC-cons}. As before, we assume an interior solution, $F_H(R - p) \in (0, 1), F_L(\phi R - p) \in (0, 1)$.

Write the Lagrangian: 

\begin{align*}
    \mathscr{L} &\equiv p \{ F_H (R - p)\lambda + F_L(\phi R - p)(1 - \lambda)\} \label{Lagrangian} \numberthis \\ 
    &- \mu(\log \phi + \log F_L(\phi R - p) - \log F_H(R - p) - \log \tau)
\end{align*}

Where: 

\begin{equation}
    \tau \equiv \frac{\lambda (1 - \underline{\lambda})}{\underline{\lambda}(1 - \lambda)}
\end{equation}

Note that $\tau$ is a constant function of exogenous parameters. 

We can then derive relevant FOCs: 

\begin{itemize}
    \item \textbf{FOC ($\phi$)}
    \begin{equation}
        \label{FOC-phi}
        \phi p R (1 - \lambda)f_L(\phi R - p) = \mu \left(1 + \frac{\phi R f_L(\phi R - p)}{F_L(\phi R - p)}\right)
    \end{equation}

    The LHS of this equation is the payoff accruing to $L$-types for any $\phi$. The RHS is something like a percentage shift in the entrance of $L$-types, scaled by payoff $\phi R$. 

    \item \textbf{FOC $p$}
    
    \begin{align*}
        0 &= F_H (R - p)\lambda + F_L(\phi R - p)(1 - \lambda) \numberthis \label{FOC-p}  \\ 
        &- p\left(f_H(R - p)\lambda + f_L(\phi R - p)(1 - \lambda) \right) \\ 
        &+ \mu \left(\frac{f_L (\phi R - p)}{F_L (\phi R - p)} - \frac{f_H(R - p)}{F_H(R - p)}\right)
    \end{align*}

    The last term is the most interesting for this paper; it captures the CRA's sensitivity to distributional effects (i.e., the fact that the CRA cares not only \emph{how many} firms are buying ratings, but about the types of those firms, since that impacts the allowable amount of lying, etc.)

    The first two terms are standard monopoly pricing objects (the first term says that raising prices gets us more per transaction, the second that we lose firms at the margins by doing so.)

    \item \textbf{Complementary Slackness}
    
    \begin{equation}
        \label{slackness}
        \mu(\log \phi + \log F_L(\phi R - p) - \log F_H(R - p) - \log \tau) = 0
    \end{equation}

    This is \eqref{MC-cons}, but log-linearized for convenience.
\end{itemize}

\subsection{Expression for Lagrange Multiplier}

It's possible to rewrite the FOC ($\phi$), \eqref{FOC-phi}, to get an expression for $\mu$: 

\begin{equation}
    \label{mu}
    \mu = \underbrace{\phi R(1 - \lambda)F_L(\phi R - p)}_{\text{total profits for $L$-types}} \cdot \frac{f_L(\phi R - p)}{F_L(\phi R - p) + \phi R f_L (\phi R - p)}
\end{equation}

Note that that the multiplier (which corresponds roughly to the value of loosening the credibility constraint) depends only on properties of low types. This is because the dishonesty which attracts low types has no effect on high types. 

\newpage

\subsection{Analytical Results}

In this section, we show some properties of the general model.

\begin{proposition}[Existence]
    \label{existence}
If $F_H, F_L \in C^0$, the CRA's optimization problem has a solution.
\end{proposition}

\begin{proof}
WLOG, we can restrict our attention to the subset $(\phi, p) \in X \equiv \{[0, 1] \times [0, R]\}$ of $\mathbb{R}^2$ (if the price exceeded $R$, nobody would enter.)

The feasible set of the optimization problem is: 

\begin{align}
    \label{set-C}
    C \equiv \{(\phi, p) | g(\phi, p) \leq 0\} \cap X \\ 
    g(\phi, p) \equiv  \phi F_L(\phi R - p) - \tau F_H (R - p)
\end{align}

By Heine-Borel, $C \subset \mathbb{R}^2$ is compact if it is closed and bounded. 

\begin{itemize}
    \item \underline{Bounded:} The box $[0, 1] \times [0, R]$ is clearly bounded. $C \subseteq X$, so $C$ is also bounded.

    \item \underline{Closed:} $X$ is also trivially closed. The finite intersection of closed sets is closed, so we need to show that $\{(\phi, p) | g(\phi, p) \leq 0\}$  is closed. 
    
    This is the preimage of a closed set under some map $g$. If $g$ is a continuous map, then the set is also closed. The RHS of $g$ is clearly continuous jointly in $(\phi, p)$. The LHS is a continuous function $F_L$ composed with a linear map $\phi R - p$, multiplied by a projection $(\phi, p) \mapsto \phi$. So it's continuous as well, which implies $g$ is.\footnote{The reason to be pedantic about this is that it is quite possible for functions to be continuous separately in each variable, but not be a continuous map.}
\end{itemize}

Since the objective \eqref{CRA-objective} is continuous, we're trying to maximize a continuous function on a compact set. The Weierstrass EVT tells us this is possible.\footnote{The set $C$ is never empty, as $(0, R)$ is always in $X$ and always satisfies the inequality.}
\end{proof}

\hrule \hspace{5 em} 

\begin{proposition}[No Truth-Telling] If $F_H, F_L \in C^0$, then truth-telling equilibria (where $\phi = 0$) are strictly dominated, and there is always some low entry.
\end{proposition}

\begin{proof}
    If $\phi = 0$, then the CRA is solving a standard monopoly pricing problem: 

    \[ \max\limits_{p \in [0, R]} p F_H(R - p) \]

    At the corners, profits are zero, so as long as $\min \supp F_H < R$, there will be some entry. This means that $g(0, p^*) < 0$.

    But $\phi F_L(\phi R - p^*)$ is continuous in $\phi$. So it's feasible to increase $\phi$ (subject to the same condition on $F_L$) such that there is an infinitesimal amount of low entry, which is strictly better than the alternative. 
\end{proof}

\newpage

\begin{proposition}[Constraint Binds] If $F_H, F_L \in C^1$, and the following conditions hold, then the market credibility constraint in \eqref{MC-cons} will always bind.

\begin{enumerate}
    \item The market reward $R$ is nonzero.
    \item The share of low types $1 - \lambda$ is nonzero.
    \item The density $f_L(\phi R - p)$ is nonzero within the feasible region (e.g., the support is connected.)
\end{enumerate}

\end{proposition} 

\begin{proof}
    The fact that $F_H, F_L \in C^1$ means that the objective and constraint are each differentiable, which justifies the use of the Lagrangian \eqref{Lagrangian}.\footnote{Subject to some regularity conditions.} We can solve for the Lagrange multiplier, $\mu$ (given in \eqref{mu}):

    \[     \mu = \underbrace{\phi R(1 - \lambda)F_L(\phi R - p)}_{\text{total profits for $L$-types}} \cdot \frac{f_L(\phi R - p)}{F_L(\phi R - p) + \phi R f_L (\phi R - p)} \]
    
    When this multiplier is nonzero, we know that the complementary slackness constraint \eqref{slackness} reduces to the credibility constraint \eqref{MC-cons}.

    The conditions given above hold iff the multiplier is strictly positive.

    Note that we know that $F_L(\phi R - p) > 0$ (there is some low entry) and $\phi > 0$ by the preceding proposition.
\end{proof}


\hrule \hspace{5 em} 

The intuition for this next result is to analyze what happens when the ``private debt market'' is able to distinguish between high and low types.

One weak way of sorting is first-order stochastic dominance (FSD), which says that $\forall x \in \mathbb{R}$, high types don't have a lower chance of doing better than $x$, compared to low types.

\begin{proposition}[Strict FSD] If $F_H, F_L \in C^0$ and $F_H(x) \leq F_L(x)$ everywhere, then RESULTS.
\end{proposition}

\begin{proof}
\end{proof} 

\newpage

\section{Numerical Implementation}

The code and output for what follows is made available on GitHub, at \url{https://github.com/arnavs/credit-pricing}, under a permissive BSD-3 license.

\subsection{Overview}

We implemented the model in Julia (version 1.1.0.) The basic approach was to write a function \texttt{utility([$\phi$, $p$]; model = $m$)} which returns the CRA payoff for a point in the feasible set, and 0 otherwise.

We fed this function to the \href{https://github.com/robertfeldt/BlackBoxOptim.jl}{\color{blue} BlackBoxOptim.jl} optimization package, which is designed for use without gradient information from the objective. Having the optimizer run for 20 seconds per model was sufficient to find an approximate optimum. 

Note that this choice was for convenience; Julia has capable autodifferentiation tools, and in any case analytic gradients for our objective are possible for ``well-behaved'' distributions $F_L, F_H$.

\newpage 

\section{Uniform Case}

\subsection{Analytical Results}

\subsection{Numerical Results and Comparative Statics}

\newpage

\section{Conclusion}

In this note, we present and discuss a simple model where a credit rater jointly chooses prices $p$ and dishonesty $\phi$ in a one-shot game. We find that the CRA obeys a standard monopoly pricing rule, modified to account for sensitivity to the \emph{quality} (and not just quantity) of customers. In particular, the CRA cares about the marginal shift in fractions of high and low type firms brought on by pricing changes.

The other chief results are that the credibility constraint \eqref{MC-cons} binds always (subject to loose restrictions), and that {SOMETHING FOLLOWS FROM FSD.}

We also created a \href{https://github.com/arnavs/credit}{\color{blue} GitHub repository} with Julia code and results for a baseline case of the model (uniform, i.i.d outside options), which we paid special attention to in this note. \\ 

The problem of balancing a credit rater's need for economic incentives, against the public's need for reliable and undistorted information, is a significant policy challenge. We hope that this note is able to contribute to that discussion.

\newpage

\section{References}

\newpage

% \appendix
% \addcontentsline{toc}{section}{Appendices}
% \section{Computational Appendix}

\end{document}
